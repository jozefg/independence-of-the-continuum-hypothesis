\documentclass[12pt]{amsart}
\usepackage{amsmath, amssymb, amsopn, stmaryrd, amsthm}
\usepackage{microtype, tgpagella}
\usepackage{tikz-cd}
\usepackage{fullpage}
\usepackage[numbers]{natbib}
\usepackage[hyperfootnotes=false]{hyperref} % Who the hell wants hyperlinked footnotes
\usepackage{parskip}

\renewcommand{\epsilon}{\varepsilon}
\newcommand{\pto}[1]{\rightharpoonup}
\newcommand{\den}[1]{\llbracket #1 \rrbracket}
\newcommand{\cat}[1]{\ensuremath{\mathbf{#1}}}
\newcommand{\sheaves}[2][]{\ensuremath{\mathbf{Sh}_{#1}(#2)}}
\newcommand{\presheaf}[1]{\ensuremath{\widehat{#1}}}
\newcommand{\Ccat}{\ensuremath{\mathbb{C}}}
\newcommand{\Dcat}{\ensuremath{\mathbb{D}}}
\newcommand{\Etop}{\ensuremath{\mathcal{E}}}
\newcommand{\Ftop}{\ensuremath{\mathcal{F}}}
\newcommand{\sheafify}{\ensuremath{\mathbf{a}}}
\newcommand{\yoneda}{\ensuremath{\mathbf{y}}}
\newcommand{\pow}[1]{\ensuremath{\mathcal{P}\left(#1\right)}}
\newcommand{\powfin}[1]{\ensuremath{\mathcal{P}_{fin}\left(#1\right)}}

\newcommand{\epi}{\ensuremath{\twoheadrightarrow}}
\newcommand{\mono}{\ensuremath{\rightarrowtail}}
\newcommand{\im}[1]{\ensuremath{\mathrm{Im}(#1)}}
\DeclareMathOperator{\charmap}{char}
\DeclareMathOperator{\true}{true}
\DeclareMathOperator{\sub}{Sub}
\DeclareMathOperator{\clsub}{ClSub}
\DeclareMathOperator{\dom}{Dom}
\DeclareMathOperator{\cod}{Cod}
\DeclareMathOperator{\epis}{Epi}
\DeclareMathOperator{\into}{in}
\DeclareMathOperator{\inl}{inl}
\DeclareMathOperator{\inr}{inr}


\newtheorem{thm}{Theorem}[section]
\newtheorem{cor}[thm]{Corollary}
\newtheorem{lem}[thm]{Lemma}
\newtheorem{remark}[thm]{Remark}
\newtheorem{example}[thm]{Example}
\newtheorem{defn}[thm]{Definition}

\title{On the Independence of the Continuum Hypothesis}
\author{Daniel Gratzer}
\date{\today}

\begin{document}
\begin{abstract}
  In these notes we present a short, mostly self-contained proof of
  the independence of the continuum hypothesis. The development is
  topos-theoretic: we shall be presenting a specific topos which can
  model ZFC for which the continuum hypothesis fails. This
  demonstrates the connection between the logical methods of forcing,
  Beth semantics and Grothendieck toposes. This development closely
  follows that given in~\citet[Chapter~6]{MacLane:92}, a
  simplification of the original proof given in~\citet{Cohen:63}. With
  these notes I hope to illustrate the elegant topos theoretic
  approach to forcing results that has become increasingly common in
  logic, type theory, and mathematics at large. The concluding
  section~\ref{sec:conclusion} contains a number of interesting and
  related developments for the curious reader.
\end{abstract}
\maketitle

\section{Background}\label{sec:background}

These notes are not meant to serve as a complete introduction to topos
theory. Therefore, the background section of these notes, rather than
being the first 5 chapters of~\citet{MacLane:92}
or~\citet{Johnstone:14} will contain more or less an accumulation of
definitions and lemmas that we will need. These will be more useful
for ensuring that I have things to reference than for the reader to
learn. It also comes with the moderate advantage that I get to inflict
the my peculiarities upon you dear reader.

We begin by defining the notion of an \emph{elementary topos}.
\begin{defn}\label{defn:background:topos}
  An elementary topos $\Etop{}$ is a category that
  \begin{itemize}
  \item has all finite limits
  \item is cartesian closed
  \item has a subobject classifier
  \end{itemize}
\end{defn}
The topos that we are interested in will be ones that satisfy certain
characteristics making them into a model of ZFC. Modeling the axiom of
choice, the law of the excluded middle, and the existence of infinite
sets in particular present challenges. The next set of definitions are
the categorical analogs of these traits.
\begin{defn}\label{defn:background:nno}
  A natural number object (NNO) is an object $N \in \Ccat$ with arrows
  $s : N \to N$ and $z : 1 \to N$ so that for any object $A \in \Ccat$
  with $f : 1 \to A$ and $g : A \to A$, there exists a unique $h$ so
  that
  \[
    \begin{tikzcd}
      1 \ar[r, "z"] \ar[dr, swap, "f"] & N \ar[d, dashed, "h"] \ar[r, "s"]& N \ar[d, dashed, "h"]\\
      & A \ar[r, "g"]& A
    \end{tikzcd}
  \]
\end{defn}

\begin{example}
  In the category of sets, $\cat{Set}$, the set of all natural numbers
  $\mathbb{N}$ forms a natural number object.
\end{example}

\begin{example}
  In any presheaf $\presheaf{\Ccat}$, the constant presheaf $A \mapsto
  \mathbb{N}$ forms a natural number object.
\end{example}

\begin{example}
  NNOs are reflected by geometric morphism. In particular, any
  reflective subtopos inherits an NNO from the full topos.
\end{example}

Having categorified the definition of the set of natural numbers, we
now turn to the logical aspects of a topos. The internal logic of a
topos is, in general, intuitionistic and thus validates neither the
law of the excluded middle nor the axiom of choice. We shall need both
of these for the topos we're constructing. Therefore, we turn to
defining what toposes \emph{do} satisfy these principles.

\begin{defn}\label{defn:background:boolean}
  A topos is said to be boolean if the subobject classifier $\Omega$
  forms an internal boolean algebra.
\end{defn}

Booleanness can be captured in several equivalent in useful ways.

\begin{lem}\label{lem:background:boolean}
  The following conditions are equivalent
  \begin{enumerate}
  \item $\Etop$ is boolean
  \item $\neg\neg = 1 : \Omega \to \Omega$
  \item The Heyting algebra $\sub(A)$ for an $A \in \Etop$ is a
    boolean algebra
  \item $\Omega \cong 1 + 1$ with $[\true, \neg \circ \true]$.
  \end{enumerate}
\end{lem}
\begin{proof}\hfill
  \begin{itemize}
  \item $1 \iff 2$\\
    The condition that $\neg\neg a = a$ for all $a$ is equivalent to
    being boolean for any boolean algebra. $a \le \neg \neg a$ in any
    Heyting algebra, therefore all we need to show is that
    $\neg \neg a \le a$. Since $1 \le a \vee \neg a$, it suffices to
    show that $\neg \neg a \wedge (a \vee \neg a) \le a$. Since all
    Heyting algebras are distributive,
    \begin{align*}
      \neg \neg a \wedge (a \vee \neg a) &\le a \Leftrightarrow \\
      (\neg \neg a \wedge a) \vee (\neg \neg a \wedge \neg a) &\le a \Leftrightarrow \\
      a \vee \bot &\le a \Leftrightarrow \\
      a &\le a
    \end{align*}
    For the other direction, it then suffices to show that
    $1 \le \neg \neg (a \vee \neg a)$
    \begin{align*}
      1 &\le \neg \neg (a \vee \neg a) \Leftrightarrow\\
      \neg (a \vee \neg a) &\le 0 \Leftrightarrow\\
      \neg a \wedge \neg \neg a &\le 0 \Leftrightarrow\\
      0 &\le 0 \Leftrightarrow
    \end{align*}
    so we're done.
  \item $1 \iff 3$\\
    We know that the Heyting algebra $\sub(A)$ for any $A \in \Etop$
    corresponds naturally to $\hom(A, \Omega)$. This is a boolean
    algebra precisely when $\Omega$ forms one internally. For the
    reverse direction, if we have $\neg\neg = 1$ in every subobject
    preorder, $\sub(A)$, naturally in $A$, then by Yoneda this holds
    internally to $\Omega$.
  \item $1 \iff 4$\\
    For this, let us first show that $4 \implies 1$. If
    $i = [true, \neg \circ true] : 1 + 1 \cong \Omega$, then it is clear
    that
    \[
      \begin{tikzcd}
        1 + 1 \ar[r, "i"] \ar[d, "tw"] & \Omega \ar[d, "\neg"] \\
        1 + 1 \ar[r, "i"] & \Omega
      \end{tikzcd}
    \]
    commutes where $tw : 1 + 1 \to 1 + 1$ is the canonical twisting
    map. It satisfies the properties that $tw \circ \inl = \inr$ and
    vice versa. Now this means that $i \circ tw \circ i^{-1} = \neg$
    but then
    \[
      \neg\neg = i \circ tw \circ i^{-1} \circ i \circ tw \circ i^{-1} =
      i \circ tw \circ tw \circ i^{-1} = 1
    \]
    so we indeed have that $\Omega$ is an internal boolean
    algebra. Next, we must show that $1 \implies 4$. We know that
    $\true \wedge \neg \true = 0$ so
    $\true \vee \neg \true = \true + \neg \true$ as subobjects of
    $\Omega$. Let us call $m = \true \vee \neg \true$, then
    pictorially, we have that
    \[
      \begin{tikzcd}
        1 + 1 \ar[ddrr, dashed, swap, "m"] & &
        1 \ar[dd, "\true"] \ar[ll]\\
        &&\\
        1 \ar[rr, "\neg \true"] \ar[uu] & & \Omega\\
      \end{tikzcd}
    \]
    However, we know that since $\Omega$ is boolean,
    $a \vee \neg a = 1$. Therefore, $\true \vee \neg \true = 1$. This
    tells us that $1 + 1 \cong \Omega$ as required.
  \end{itemize}
\end{proof}

For our purposes of using a topos to model ZFC, booleaness will be
essential. A boolean topos will validate the law of excluded
middle\footnote{Exercise, show that in a boolean topos
  $\den{\forall x.\ x \vee \neg x}$ holds} which is crucial for
validating the rules of ZFC. There then arises the natural question of
taking an existing topos and modifying it so that it is boolean. This
is easily done using a Lawvere-Tierney topology

\begin{defn}\label{defn:background:lawveretierney}
  A Lawvere-Tierney Topology is a map $j : \Omega \to \Omega$ so that
  \begin{enumerate}
  \item $j \circ \true = \true$
  \item $j \circ j = j$
  \item $\wedge \circ (j, j) = j \circ \wedge$
  \end{enumerate}
\end{defn}
While the subject of Lawvere-Tierney topologies on an elementary topos
is a rich one, we shall limit ourselves to the essentials for our
purposes here. It will be important to know that given a
Lawvere-Tierney topology we can internally define a notion of
``sheaf'' which matches up to that of a sheaf on a Grothendieck
topology. In order to do this, first we define
\begin{defn}\label{defn:background:closure}
  For a topology $j$ on $\Etop$, the closure of $A \mono B$ is the
  subobject of $B$ classified by $j \circ \charmap(A)$, $\bar{A}$. A
  subobject is said to be closed if $\bar{A} = A$.
\end{defn}

We can define an alternative subobject classifier which internalizes
a Lawvere-Tierney topology as
\[
  \begin{tikzcd}
    \Omega_j \ar[r, "i", dashed] &
    \Omega \ar[r, shift left=0.5em, "1"]
    \ar[r, shift right=0.3em, swap, "j"] &
    \Omega
  \end{tikzcd}
\]
This has the nice property that it classifies closed subobjects.
\begin{lem}\label{lem:background:closedfactors}
  If $A \mono B$ is a closed subobject under $j$, then $\charmap(A)$
  factors through $\Omega_j$.
\end{lem}
\begin{proof}
  Suppose that $A$ is a closed subobject, then we know that
  $\charmap(A) \circ j = \charmap(A)$ by definition. Therefore,
  \[
    \begin{tikzcd}
      A \ar[d, dashed] \ar[dr, "\charmap(A)"] & &\\
      \Omega_j \ar[r]&
      \Omega \ar[r, shift left=0.3em, "1"] \ar[r, shift right = 0.3em, swap, "j"] & \Omega
    \end{tikzcd}
  \]
  by the universal property of an equalizer.
\end{proof}

\begin{defn}\label{defn:background:dense}
  A subobject $A \mono B$ is said to be dense if $\bar{A} \cong B$
\end{defn}

We are now in a position to define the internalization of sheaves. The
inspiration for this definition comes from the fact that in a
Grothendieck topology, a sieve is a dense subfunctor of
$\yoneda(C)$. We therefore can define sheaves purely in terms of dense
subobjects.

\begin{defn}\label{defn:background:sheaves}
  An object $F \in \Etop$ is said to be a sheaf if for every dense
  subobject $A \mono B$, a morphism $f : A \to F$ has a unique
  extension $g : B \to F$ and every $g : B \to F$ corresponds to a
  unique $f : A \to F$. The former is captured by the diagram
  \[
    \begin{tikzcd}
      A \ar[d, rightarrowtail, swap, "m"] \ar[r, "f"] & F\\
      B \ar[ur, dashed, swap, "g"]
    \end{tikzcd}
  \]
\end{defn}

\begin{lem}\label{lem:background:denseclosediso}
  For any dense monomorphism $m : A \mono B$. Pullback along $m$ induces an isomorphism between the
  closed subobjects of $B$ and the closed subobjects of $A$: $\clsub(A) \cong \clsub(B)$.
\end{lem}
\begin{proof}
  Suppose we have some closed subobject $U \mono B$. We know that
  $\overline{m^{-1}(U)} = m^{-1}(\overline{U}) = m^{-1}(U)$ because closure is natural. Therefore,
  $m^{-1}$ at least induces a map $\clsub(B) \to \clsub(A)$.

  Next, let us consider $U \mapsto \overline{\exists_m U} : \clsub(A) \to \clsub(B)$. We will prove
  that it is inverse to $m^{-1}$:
  \[
    m^{-1}(\overline{\exists_m U}) = \overline{m^{-1}(\exists_m U)} = \overline{U} = U
  \]
  This tells us that for any monomorphism that this is at least a partial inverse (recall that
  $\exists_m$ is a simply post-composition with $m$ and therefore $m^{-1} \circ \exists_m = 1$ for
  any monomorphism). For the reverse, suppose we have some closed $U \mono B$. Observe that
  $\exists_m m^{-1}(U)$ is the image of $m^{-1}(U)$, which is equivalently phrased as $U \cap
  A$. Therefore:
  \[
    \overline{\exists_m m^{-1}(V)} =
    \overline{A \cap U} =
    \overline{A} \cap \overline{U} =
    B \cap U =
    U \qedhere
  \]
\end{proof}
\begin{lem}\label{lem:background:omegajsheaf}
  $\Omega_j$ is a $j$-sheaf.
\end{lem}
\begin{proof}
  Suppose we have a dense map $m : A \mono B$ and $f : A \to \Omega_j$, we wish to construct
  $g : B \to \Omega_j$ so that $g \circ m = f$. We know that
  $i \circ f : A \to \Omega_j \to \Omega$ corresponds to a closed subobject of $A$. We apply
  Lemma~\ref{lem:background:denseclosediso} to construct a unique closed subobject,
  $i \circ m' : B \to \Omega_j \to \Omega$ such that $i \circ m' \circ m = i \circ f$.
  Now, since $i$ is a monomorphism (it is the equalizer) so $m' \circ m = f : A \to \Omega_j$.
  We know that $m'$ is unique with this property so we are done.
\end{proof}
Sheaves enjoy several nice properties with regards to closure which
will be useful for establishing some facts later on.
\begin{lem}\label{lem:background:sheavesclosed}
  For any $A \mono B$ if $A$ is a sheaf then $A$ is closed.
\end{lem}
\begin{proof}
  We know that by definition that the mono $m : A \mono \bar{A}$ is
  dense and since $A$ is a sheaf we must have
  \[
    \begin{tikzcd}
      A \ar[d, rightarrowtail, "m"] \ar[r, equals] & A\\
      B \ar[ur, dashed]
    \end{tikzcd}
  \]
  so in particular $m$ must be an isomorphism.
\end{proof}

Finally, while the proof is too far out of scope for these notes, it
is important to note that sheaves on a topology $j$ form a full
subcategory of a topos, $\Etop$. This category, denoted
$\sheaves[j]{\Etop}$, is in fact a reflective subcategory with
\[
  \begin{tikzcd}
    \sheaves[j]{\Etop} \ar[rr, bend left, "\iota"] & \top &
    \makebox[0pt]{\phantom{\textbf{Sh}}$\Etop$}\phantom{\sheaves[j]{\Etop}}
    \ar[ll, bend left, "\sheafify"]
  \end{tikzcd}
\]
The curious reader is referred to either~\citet{MacLane:92}
or~\citet{Johnstone:14}. Additionally Jon Sterling gave a talk on this
and ought to distribute his notes. Please go bother him if you really
want a proof. Now in our case, we shall be interested in one
particular topology, the double negation topology.
\begin{example}\label{ex:background:doublenegation}
  For any topos, $\neg\neg : \Omega \to \Omega$ forms a topology.
\end{example}
The useful property of the double-negation topology for our purposes
is that $\sheaves[\neg\neg]{\Etop}$ will always form a boolean
subtopos of $\Etop$. Later on, we will use this to construct a boolean
topos out of a topos modeling the forcing construction we wish to
implement.

\begin{lem}\label{lem:background:notnotboolean}
  $\sheaves[\neg\neg]{\Etop}$ is boolean.
\end{lem}
\begin{proof}
  By~\ref{lem:background:boolean} it suffices to show that for each
  sheaf $A$ that $\sub(A)$ is a boolean lattice. If $B \mono A$ is a
  subobject of $A$, it must be that $B$ closed as it too must be a
  sheaf by~\ref{lem:background:sheavesclosed}. But closure in under
  $\neg\neg$ means precisely that $\neg \neg B = B$ so $\sub(A)$ is
  boolean as required.
\end{proof}

An interesting aside at this point is that the double-negation
topology on a presheaf topos has a well known Grothendieck analog: the
dense topology. That is, the topology given by
\[
  J(C) = \{ S \mid \forall f : A \to C.\ \exists h.\ f \circ h \in S\}
\]
I will not prove this but will use the phrase ``dense topology'' and
``double negation topology'' interchangeably in the
notes\footnote{It's because I'm a mean-spirited person}.

We next turn to the topos-theoretic analog of the axiom of
choice. Here we are presented with two possible ways. The first is a
direct formulation of the principle from the category of sets. We wish
to generalize \emph{every surjection has a section}. We can generalize
this to a topos by replacing surjection with epimorphism to get

\begin{defn}\label{defn:background:aoc}
  A topos satisfies the axiom of choice if every $e : A \epi B$ has a
  section $s : B \to A$ so that $es = 1$
\end{defn}

However, it is often preferable to state a version termed the internal
axiom of choice but the differences are out of scope for these
notes. I will settle for merely stating it.

\begin{defn}\label{defn:background:iaoc}
  A topos satisfies the internal axiom of choice if $(-)^A$ preserves
  epimorphisms.
\end{defn}

\begin{remark}
  Any topos that satisfies the axiom of choice satisfies the internal
  version. Any topos that satisfies the internal version is
  boolean. This is due to~\citet{Diaconescu:75}.
\end{remark}

This characterization of the axiom of choice is awkward to work with
however in many toposes. A cleaner characterization can be given in
terms of the what objects \emph{generate} the topos.

\begin{defn}\label{defn:background:generate}
  A collection of objects $S$ is said to generate a category $\Ccat$
  if for any parallel arrows $f \neq g : A \to B$, there exists an
  arrow $h : C \to A$ for some $C \in S$ so that
  $fh \neq gh : C \to B$. In particular, if a topos is generated by
  $1$ and is nontrivial then it is \emph{well-pointed}.
\end{defn}

Generation by a collection of objects captures a large number of
logical principles that we will need for our topos. Most important
among these is that for a boolean Grothendieck topos generated by
subobjects of 1, we may automatically derive the validity of the axiom
of choice. This is because in a Grothendieck topos the subobject
preorders are all \emph{complete} partial orders. We demonstrate this
in the next lemma.

\begin{lem}\label{lem:background:completealgebratoaoc}
  If a boolean topos $\Etop{}$, is generated by subobjects of $1$ and
  has complete boolean algebras for subobject posets then $\Etop{}$
  satisfies the axiom of choice.
\end{lem}
\begin{proof}
  We wish to show that $e : A \epi B$ has a section $B \to A$. We then
  wish to use Zorn's lemma to find a maximal subobject $m : M \mono B$
  so that $e$ has a section $s : M \to A$. That is, so that $es = m$.
  In order to apply Zorn's lemma we wish to show that the subset of
  $\sub(B)$ is closed under chains. First we note that any such
  section is necessarily mono. This is because if $sf = sg$, then we
  must have $esf = esg$ so $mf = mg$ so $f = g$ as $m$ is mono. Now
  let us show the chain condition.

  Suppose that $(M_i)_i \subseteq \sub(B)$ so that for each $M_i$, we
  have a section $s_i$ and this forms a chain. Now we note that each
  $M_i$ is a subobject of $B \times A$ under
  $\langle m_i, s_i \rangle$. Since we assumed that $\Etop{}$ had
  complete boolean algebras for subobject lattices, let us take
  $M = \bigvee_i M_i$ taken as subobjects of $B \times A$. Now this
  tells us that $M \mono B \times A$ so we must have $m : M \to B$ and
  $s : M \to A$. We must show that $e \circ s = m$ now.

  Let us note that $A$ is a subobject of $B \times A$ under
  $\langle e, id \rangle$. Furthermore, by $s$ we have that
  $M_i \le A$ for all $i$. Therefore, by the definition of a least
  upper bound, we must have $M \le A$. Therefore, we have a mono so
  that
  \[
    \begin{tikzcd}
      M \ar[rr, rightarrowtail, "s'"] \ar[dr, swap, rightarrowtail, "{\langle m, s \rangle}"] &&
      A \ar[dl, rightarrowtail, "{\langle e, id \rangle}"]\\
      &B \times A&
    \end{tikzcd}
  \]
  this tells us that $s' = s$ so we have our desired section. Now we
  want to conclude that $m : M \to B$ is indeed a mono, however, all
  we know is that $\langle m, s \rangle$ is a mono. Therefore, we want
  to show that
  \[
    (x, x', y \mid m(x) = y \wedge m(x') = y \vdash x = x')
  \]
  in the internal logic. Since $m' = \langle m, s \rangle$ is a mono,
  we know that
  \[
    (x, x', z : B \times A \mid m'(x) = z \wedge m(x') = z \vdash x = x')
  \]
  Therefore, if we can show that
  \[
    (a, a', b \times M(b, a), M(b, a') \vdash a = a')
  \]
  then we could rewrite the assumption that $m'$ is mono to
  \[
    (x, x', b : B \mid \pi_1(m'(x)) = b \wedge \pi_1(m(x')) = b \vdash x = x')
  \]
  and since $m = \pi_1 \circ m'$ by definition, this gives us our
  desired result. We know that since each $m_i$ is mono, that if
  \[
    (x, x', b \mid m_i(x) = b \wedge m_i(x') = b \vdash x = x')
  \]
  Furthermore, this tells us that if
  \[
    (x, x', b, a, a' \mid (\langle m_i, s_1 \rangle)(x) = (b, a) \wedge
    (\langle m_i, s_1 \rangle)(x') = (b, a') \vdash x = x')
  \]
  so by functionality, since we have
  $f(x) = (b, a) \wedge f(x) = (b, a')$ we must have $a = a'$ as
  well. We know that for all $i$.
  \[
    (a, a', b \times M_i(b, a), M_i(b, a') \vdash a = a')
  \]
  Now, stepping outside the internal logic, we wish to show that
  \[
    \begin{tikzcd}
      {\pi_{12}^*(M) \wedge \pi_{13}^*(M)} \ar[dr] \ar[rr, rightarrowtail] & &
      {B \times A} \ar[ld, "{1 \times \delta}"]\\
      &{B \times A \times A}&
    \end{tikzcd}
  \]
  Let us now work with $\sub(B \times A \times A)$.
  \begin{align*}
    \pi_{12}^*(M) \wedge \pi_{13}^*(M) &= \pi_{12}^*(\bigvee_i M_i) \wedge \pi_{13}^*(\bigvee_i M_i)\\
      &= \bigvee_i \pi_{12}^*(M_i) \wedge \bigvee_i \pi_{13}^*(M_i)\\
      &= \bigvee_i \bigvee_j \pi_{12}^*(M_i) \wedge \pi_{13}^*(M_j)\\
      &= \bigvee_i \pi_{12}^*(M_i) \wedge \pi_{13}^*(M_i)\\
      &= \bigvee_i \delta\\
      &= \delta
  \end{align*}
  Where we have made use of the fact that clearly
  \[
    \bigvee_i \pi_{12}^*(M_i) \wedge \pi_{13}^*(M_i) \le
    \bigvee_i \bigvee_j \pi_{12}^*(M_i) \wedge \pi_{13}^*(M_j)
  \]
  but since $M_i$ forms a chain, supposing that $i \le j$ without loss
  of generality, we then have that
  \[
    \pi_{12}^*(M_i) \wedge \pi_{13}^*(M_j) \le \pi_{12}^*(M_j) \wedge \pi_{13}^*(M_j)
  \]
  This means that the reverse inclusion holds. All told, this means
  that we have indeed constructed an $M$ dominating the chain with a
  partial section for $e$ as required. Thus, Zorn's lemma gives us a
  maximal such subobject.

  Now suppose we have such a maximal subobject $m : M \mono B$ with a
  section $s : M \to A$. Let suppose that $M \neq A$, for it is then
  we're done. We note that $M$ must have a nonzero complement as it is
  not $B$. Since $\neg M$ is nonzero, it has two different subobjects
  of its own, $0$ and $1$. Thus $\neg M$ has two different
  characteristic maps and since subobjects of 1 generate, for some
  $V \mono 1$ we must have $t : V \to \neg M$. Consider
  \[
    \begin{tikzcd}
      A' \ar[d, "e'", twoheadrightarrow] \ar[rr] & & A \ar[d, twoheadrightarrow, "e"]\\
      V \ar[r, rightarrowtail, "t"] & \neg M \ar[r, rightarrowtail] & B
    \end{tikzcd}
  \]
  Next, we note that $V \neq 0$ because it has an arrow to a nonzero
  object. From this, it follows that $A'$ must not be 0 as $A' \epi V$
  and
  $\begin{tikzcd}
    V \ar[r, "f", shift left=0.2em] \ar[r, "g", swap, shift right=0.2em] & \Omega
  \end{tikzcd}$
  for some $f \neq g$. If $A'$ was zero then it must be that $fe = ge$
  which would give us $f = g$, a contradiction.

  Now since $A'$ is nonzero, we can once again get a subterminal
  object which maps into it, $W \mono 1$ with $w : W \to A'$. Now we
  immediately have $e'w : W \to V$ so it follows that $W \mono V$. Now
  we take the image factorization of $\Im(te'w) = e'' : W \to t(W)$.
  Now since $W \mono 1$, it must be that $e''$ is also mono and since
  $\Etop$ is balanced, we then have that $e''$ is an iso. Now as
  $t(W) \mono \neg M$, it must be that $t(W) \wedge M = 0$ so
  $t(W) \vee M = t(W) + M$. Moreover, $we''^{-1} : t(W) \to A'$ so we
  know that $t(W) \to A$ so we can form a section $M + t(W) \to A$.
  This contradicts the maximality of $M$.
\end{proof}

With this we are in a position to start our proof because a topos
which satisfies the AoC and has a NNO is powerful enough to provide a
model of ZFC that we will use to validate the independence of the
continuum hypothesis.

%%% Local Variables:
%%% mode: latex
%%% TeX-master: "main"
%%% End:

\input{overview}
\input{cohen-topos}
\section{The Preservation of Strictness}\label{sec:strictness}

Proving the strictness of the cardinal inequalities turns out to be a
rather technical endeavor. To begin with, we need to define what it
means categorically for a monomorphism to be strict. In order to do
this, we will use the internal language of the topos to define a
subobject $\epis(X, Y) \mono Y^X$.
\[
  \epis(X, Y) = \{f \mid \forall y \in Y.\ \exists x \in X.\ f(x) = y\}
\]
First we show that this actually corresponds to the epimorphisms from
$X$ to $Y$. For this we will make heavy use of the Kripke-Joyal
semantics explained in~\citet[Chapter~6]{MacLane:92}.

\begin{lem}\label{lem:strictness:epis}
  $\langle \pi_1, e \rangle : P \times X \epi P \times Y$ if and only if
  $\Lambda e : P \to Y^X$ factors through $\epis(X, Y)$.
\end{lem}
\begin{proof}
  It suffices to show that $P \Vdash \forall x.\ \exists y.\ f(x) = y$
  if and only if $\hat{f} : P \times X \to Y$ is epi. Now by the
  Kripke-Joyal semantics, $P \Vdash \forall x.\ \exists y.\ f(x) = y$
  if and only if for all $V$,
  \[
    P \times V \Vdash \exists y.\ (f\pi_1)(x) = \pi_2
  \]
  Next, this holds if and only if for some $p : U \epi P \times V$
  and some $b : U \to X$
  \[
    U \Vdash f\pi_1p(b) = \pi_2p
  \]
  Now the rule for equality simply states that this holds if the
  interpretation of these two maps are equal. That is, that
  $\epsilon \circ \langle f\pi_1p, b \rangle = \pi_2p$. However, then
  we know that
  \[
    \epsilon \circ \langle f\pi_1p, b \rangle =
    \hat{f} \circ \langle \pi_1 p, b \rangle
  \]
  here $\hat{f}$ is just the transpose of $f$. Finally, this gives us that
  $p = \langle \pi_1, \hat{f} \rangle \circ \langle \pi_1 p, b \rangle$.
  This tell us that $\langle \pi_1, \hat{f} \rangle$ is epi and this is
  true if and only if $\hat{f}$ is epi as required.
\end{proof}

Now that we have established an internal representation of the
existence of epimormorphisms we can define strict inequality.

\begin{defn}
  $X < Y$ if and only if $X \mono Y$ and $\epis(X, Y) \cong 0$.
\end{defn}

Having internalized this, we are at least now in a position to state
the theorem that we want to prove:
$X < Y \implies \sheafify(\Delta X) < \sheafify(\Delta Y)$. This
property will rely crucially on the structure of $P$. In particular we
shall show that $P$ has the Souslin property.

\begin{defn}\label{defn:strictness:souslin}
  A partial order $Q$ satisfies the Souslin property if any set of
  objects which are pairwise disjoint ($a \wedge b = 0$ for all $a$
  and $b$) is at most countable.
\end{defn}
\begin{defn}\label{defn:strictness:souslintopos}
  A topos $\Etop$ satisfies the Souslin property if it is generated by
  objects for whom $\sub(-)$ has satisfies the Souslin
  property~\ref{defn:strictness:souslin}
\end{defn}

In fact the Souslin property is precisely what we need in order to get
this fact to go through as the following lemma shows.

\begin{lem}\label{lem:strictness:souslin}
  If $X < Y$ in $\cat{Set}$ and $X$ and $Y$ are infinite. Grothendieck
  topos $\Etop$ which satisfies the Souslin property then
  $\sheafify(\Delta X) < \sheafify(\Delta Y)$.
\end{lem}
\begin{proof}
  It is clear that if $X \le Y$, then
  $\sheafify(\Delta X) \le \sheafify(\Delta Y)$. Therefore, it
  suffices to show that if $\epis(X, Y) = 0$ then
  $\epis(\sheafify(\Delta X), \sheafify(\Delta Y)) = 0$ as well. Let
  us suppose not. Then there must be a nonzero object $U$ which
  satisfies the Souslin property so that $U \to \epis(X, Y)$.
  Therefore, by lemma~\ref{lem:strictness:epis} there must be an
  epimorphism
  $g = \langle \pi_1, f \rangle : U \times X \epi U \times Y$.

  Take $x \in X$ and $y \in Y$, have two points
  $\sheafify(\Delta x) : 1 \to \sheafify(\Delta X)$ and
  $\sheafify(\Delta y) : 1 \to \sheafify(\Delta Y)$. Using these, we
  can form the two pullback squares
  \[
    \begin{tikzcd}
      V_{x, y} \ar[d, rightarrowtail] \ar[r, rightarrowtail] &
      P_y \ar[d, rightarrowtail] \ar[r, "h", twoheadrightarrow] &
      {U \cong U \times 1} \ar[d, "{(1, \sheafify(\Delta y))}"]\\
      {U \cong U \times 1} \ar[r, "{(1, \sheafify(\Delta x))}", swap]&
      U \times \sheafify(\Delta X) \ar[r, "g", twoheadrightarrow, swap]&
      U \times \sheafify(\Delta Y)
    \end{tikzcd}
  \]
  Now let us define
  $W = \{(x, y) \in X \times Y \mid V_{x, y} \neq 0\}$. First note
  that $S \cong \amalg_{x \in X} 1$. Therefore,
  $\amalg_{x \in X} 1 \times U \cong \sheafify(\Delta X) \times U$.
  Moreover, this colimit exists because $\Etop$ is a Grothendieck
  topos. Since pullbacks have a right adjoint, we know that pulling
  back along $P_y \to U \times \sheafify(\Delta X)$ gives us that
  $\amalg_x V_{x, y} \to P_y$. However, since
  $\amalg_x U \times 1 \cong U \times \sheafify(\Delta X)$ we know
  that this isomorphism is preserved by pullback so in fact
  $\amalg_x V_{x, y} \cong P_y$. Now since $U$ is known to be nonzero
  and we have an epimorphism $P_y \epi U$, it must be that $P_y$ is
  also nonzero. Therefore, we know that there is some $x, y$ so that
  $V_{x, y}$ is nonempty since $\amalg_x V_{x, y} \cong P_y$. This
  tells us for every $y \in Y$ there exists an $x \in X$ so that
  $(x, y) \in W$.

  This tells us that $\pi_2 : W \to Y$ is a surjection of sets so it
  suffices to show that $X \epi W$ in order to show our contradiction
  that $X \epi Y$. Now, in order to this, let us first note that
  $V_{x, y} \mono U$. Moreover, if $y \neq y'$, then it must be that
  $V_{x, y} \wedge V_{x, y'} = 0$. This is because $y : 1 \mono Y$ and
  $y' : 1 \mono Y$ clearly are disjoint subobjects of $Y$ in
  $\cat{Set}$. However, this pullback diagram is preserved by $\Delta$
  since limits are computed pointwise and then by $\sheafify$ since it
  is left exact. Therefore,
  $\sheafify(\Delta(y)) \wedge \sheafify(\Delta(y')) = 0$. Finally,
  since meets are preserved by pullback and so is $0$, this gives us
  that $V_{x, y} \wedge V_{x, y'} = 0$.

  Now it is time to make use of this Souslin property. We know that
  $W_x = \{y \mid (x, y) \in W\}$ must be at most countable as they
  are necessarily disjoint. Since $S$ is assumed to be infinite, since
  we know that
  \[
    \left\vert W \right\vert = \left\vert S \right\vert \times \omega
    = \left\vert S \right\vert
  \]
  Therefore, $S \cong W$ and we have our desired surjection.
\end{proof}

Having proving all of this, all that remains is to show that our $P$
does in fact satisfy the Souslin property. This turns out to be a
fun\footnote{Boring.} exercise in order theory.

\begin{lem}
  $\sheaves[\neg\neg]{P}$ satisfies the Souslin property.
\end{lem}
\begin{proof}
  We know that $\yoneda(p)$ generates $\sheaves[\neg\neg]{P}$. It is
  clear that if $A \mono B$ and $B$ has the Souslin property, then so
  does $A$ (as $\sub(A) \subseteq \sub(B)$). Therefore, since
  $\yoneda(p) \mono 1$, it suffices to show that $1$ has the Souslin
  property. Therefore, suppose that $(U_i)_i$ is a family of pairwise
  disjoint nonzero subterminals. We wish to show that it is at most
  countable. Now we know that $\yoneda(p_i) \le U_i$ and that
  $U_i \wedge U_j = 0$ so $\yoneda(p_i) \wedge \yoneda(p_j) =
  0$. Therefore, $(U_i)_i$ really represents a set of pairwise
  incompatible conditions. We wish to show that this is at most
  countable.

  Let us define
  \[
    A_i = \{p_i \mid p_i \text{ is defined for $n$ entries}\}
  \]
  We wish to show by induction on $i$ that each $A_i$ is
  countable. Since $\bigcup_i A_i = (p_i)_i$ this shows our original
  goal. Suppose that $A_i$ is countable for all $i < j$, we wish to
  show that $A_j$ is countable. To show that $A_j$ is countable, it
  suffices to show that $A_{j, n}$ is countable where
  \[
    A_{j, n} = \{p_i \mid p_i \in A_j \wedge \exists b.\ p_i(b, n) \downarrow\}
  \]
  this is because $\bigcup_{i \in \mathbb{N}} A_{j, i} = A_j$. Now we
  can divide each $A_{j, n}$ into two sets, $A_{j, n, 0}$ and $A_{j, n
    1}$ where
  \[
    A_{j, n, i } = \{p_i \mid p_i \in A_j \wedge \exists b.\ p_i(b, n) = i\}
  \]
  However, we note that  $A_{j, n, i}$ must be comprised of pairwise
  incompatible counditions still. Since we know that for any $p, q \in
  A_{j, n i}$ that $p(b_p, n) = q(q_p, n)$, it must be that there is
  some other $b', n'$ so that $p(b', n') \neq q(b', n')$. Therefore,
  the set
  \[
    R_{j, n, i } = \{p_i \setminus \{(b_{p_i}, n, i)\} \mid p_i \in A_{j, n, i}\}
  \]
  is pairwise incompatible. Since it is comprised of conditions of
  length $j - 1$, it must be that $R_{j, n, i} \subseteq A_{j - 1}$ so
  it is countable. Furthermore, then $A_{j, n, i}$ is countable and so
  is $A_{j, n}$ as we required. Therefore, $A_j$ is countable and we
  are done by induction.
\end{proof}
Now all told, this gives us that there is no epimorphism
$\sheafify(\Delta \Omega^N) \epi \sheafify(\Delta B)$ nor an
epimorphism $\sheafify(\Delta N) \epi \sheafify(\Delta \Omega^N)$ so
that
\[
  \begin{tikzcd}
    \sheafify(\Delta N) \ar[r, rightarrowtail, "\sheafify(\Delta(\iota_1))"] &
    \sheafify(\Delta \Omega^N) \ar[r, rightarrowtail, "\sheafify(\Delta(\iota_2))"] &
    \sheafify(\Delta B) \ar[r, rightarrowtail, "m"] &
    \Omega_{\neg\neg}^{\sheafify(\Delta B)}
  \end{tikzcd}
\]
indeed has strict inclusions for the first two maps. Thus, we have
established an object which lies strictly between
$\sheafify(\Delta N)$ and $\Omega_{\neg\neg}^{\sheafify(\Delta N)}$.
This, combined with the result of~\citet{Fourman:80} is sufficient to
establish the independence of the continuum hypothesis from ZFC.

\section{Forcing in a More General Context}\label{sec:conclusion}

Having proven the independence of the continuum hypothesis, I wanted
to take a little time to discuss how this proof fits into a broader
context. In general forcing is incredibly useful for establishing
independence results in both set theory and type theory. This proof
shows how syntactic forcing proofs can be smoothly translated into a
proof about toposes. I am not capable of speaking of interesting
results established in set theory using forcing but several
developments in type theory have used a topos-theoretic forcing
technique. Specifically,~\citet{Coquand:12} represents a coherent
introduction to some of the developments done in~\citet{Coquand:04}
and~\citet{Coquand:16}. This work was developed in
the~\citet{Jaber:14}. While not directly using topos-theoretic
forcing,~\citet{Escardo:13} and~\citet{Sterling:16} are both results
about type theory done by a similar technique.

The alternative characterization of forcing in terms of boolean valued
models was explored by Scott and others during the 60s and 70s. These
are remarkably topos-theoretic semantics in which the validity of a
statement isn't a simple boolean but given in terms of an element of a
complete boolean lattice. By quotienting this lattice by a particular
ultrafilter one can translate forcing proofs into this framework. I am
not well positioned to recommend literature on this
but~\citet{Jech:08} contains an approachable introduction and our own
Clive Newstead has produced notes on this~\cite{Newstead:12}.

More familiar to logicians will be Kripke semantics and Beth
semantics. In these we index the $\Vdash$ relation with a ``world'' at
which we consider it. These worlds are assumed to form a preorder
which is intended to represent time with $w_1 \le w_2$ implying that
$w_2$ is a possible future of $w_1$. Kripke semantics correspond
closely to presheaf semantics on the poset of worlds. Beth semantics
add the local character sheaves enjoy to the semantics. Accordingly
then, Kripke semantics, step-indexed logical relations, and Beth
semantics in general can be translated as a special case of the
Kripke-Joyal semantics for the internal logic of some presheaf or
sheaf topos. This idea for intuitionistic logic was discussed
in~\citet{Fourman:82} and~\citet{Fourman:13}. Kripke and Beth
semantics are given a lengthy consideration in~\citet{Dummett:00}
and~\citet{Troelstra:88}. Recently, the more topos-theoretic approach
that has been present in developments like~\citet{Dreyer:09} has been
made explicit in the work of~\citet{Birkedal:11}.


\bibliographystyle{plainnat}
\bibliography{sheaves-and-forcing}{}
\end{document}
